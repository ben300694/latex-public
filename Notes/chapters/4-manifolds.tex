% % % % % % % % % % % % % % % % % % % % % % % % %
\chapter{4-Manifolds}
% % % % % % % % % % % % % % % % % % % % % % % % %

% ----------------------
\section{Bordism}
% ----------------------

\begin{definition}
	The $n$-dimensional oriented \textit{cobordism group}
	\index{cobordism group}
	over the space $X$ is
	\begin{equation*}
		\Omega_{n}[X] =
			\frac{ \{ f \colon M^{n} \rightarrow X \mid M^n \textrm{ oriented, closed, } 
			n\textrm{-dim. manifold} \}}{\textrm{bordism}}
	\end{equation*}
	\marginnote{If $\alpha \colon M^{n} \rightarrow X$
		represents a bordism class, $M^n$ is allowed to
		have more than one component.}
\end{definition}

\begin{proposition}[{\citep[13.15, p. 319]{kauffman1987knots}}]
	\begin{itemize}
		\item Pushing forward a fundamental class
		\begin{align*}
			\Omega_n [X] & \rightarrow H_{n}(X ; \Z) \\ 
			[f \colon M^{n} \rightarrow X] & \mapsto f_{*}([M])
		\end{align*}
		is an isomorphism for $n \le 3$.
		
		\item The sequence
		\begin{equation*}
			\Omega_{4}[\ast] \rightarrow \Omega_{4}[X] \rightarrow H_{4}(X ; \Z)
		\end{equation*}
		is exact.
	\end{itemize}
\end{proposition}

%TODO

% ------------------------------------
\section{Mazur manifolds}
% ------------------------------------

\textbf{References}:
\begin{itemize}
	\item Homology spheres: \citep{saveliev2013invariants}
	\item \citep{akbulut1979mazur}
\end{itemize}

%TODO
TODO


% ------------------------------------
\section{Andrews-Curtis Conjecture}
% ------------------------------------

\citep{SCP4}

\begin{definition}
	A \textit{balanced presentation} \index{presentation!balanced} is a presentation
	\[
		\langle g_1, \ldots, g_n \mid r_1, \ldots, r_n \rangle
	\]
	with the same number of generators and relations.
	
	The Andrew-Curtis moves on a balanced presentation are
	\begin{enumerate}[label=(\roman*)]
		\item \textbf{1-handle slides:} Replace a pair of generators
		$ \{ x, y \} $ by $ \{ x, xy \} $
		
		\item \textbf{2-handle slides:} Replace a pair of relations
		$ \{ r, s \} $ by $ \{ grg^{-1}s, s \} $, where $ g $ is any word in the generators
		
		\item \textbf{1-2-handle cancellations:} Add a generator together with
		a new relation killing it
	\end{enumerate}
	In particular, you are \underline{not} allowed to make a copy of a relation to
	keep for later use (which would correspond to 
	\textbf{2-3-handle creation/cancellation}).
\end{definition}

\begin{conjecture*}[Andrews-Curtis conjecture \textcolor{red}{(False?)}]
	Any balanced presentation of the trivial group can be transformed
	by Andrews-Curtis moves to the trivial presentation.
\end{conjecture*}

\begin{example}
	%TODO
	TODO
	\[
		\langle x, y \mid xyx = yxy, x^{5} = y^{4} \rangle
	\]
	is a balanced presentation of the trivial group,
	but until now nobody was able to find a sequence of
	Andrews-Curtis moves to transform it into the trivial presentation
	\[
		\langle x, y \mid x, y \rangle
	\]
\end{example}

%TODO
TODO
Explain how to construct homotopy $4$-spheres from balanced
presentations of the trivial group
\citep{akbulut1985potential}

\marginnote{
	In \citep[I.3]{kirby2006topology} Kirby suspected that the
	\textit{Dolgachev surface}
	\index{Dolgachev surface} $E(1)_{2,3}$ 
	(an exotic copy of the rational elliptic surface
	$ \CP{2} \# 9 \overline{\CP{2}} $)
	might require 1- and/or 3-handles.
	But in \citep{akbulut2008dolgachev} Akbulut found a
	handlebody presentation without those.
}
\begin{openquestion*}[\textcolor{olive}{(Still open?)}]
	Does a simply connected, closed, smooth $4$-manifold need
	$1$-handles and/or $3$-handles?
\end{openquestion*}

\begin{observation}
	If exotic $\sphere{4}$ exist, their handle decomposition must
	contain $1$- or $3$-handles.
	
	For this suppose you have a handle decomposition of a manifold
	with the homology of $\sphere{4}$ and no $1$- and $3$-handles,
	then there could not be any $2$-handles either because these would
	give nontrivial second homology. 
\end{observation}

\begin{definition}
	Use the following notation to denote the abelian group
	\begin{equation*}
		\Gamma_n = 
		\frac{\textrm{orientation preserving diffeomorphisms of } \sphere{n-1}}
			 {\textrm{those that extend to a diffeomorphism of } \disk{n}}
	\end{equation*}
\end{definition}
\marginnote{
	\begin{itemize}
		\item $\Gamma_{1} = \Gamma_{2} = 0$
		\item Munkres, Smale: $\Gamma_{3} = 0$
		\item Cerf: $\Gamma_{4} = 0$
		\item Kervaire, Milnor: $\Gamma_{5} = \Gamma_{6} = 0$,
		$\Gamma_{7} = \Z / 28$
	\end{itemize}
	}

\begin{proposition}
	For $n \ge 5$ we can identify $\Gamma_{n}$
	with the set of oriented smooth structures
	of the topological $n$-sphere.
	I.e. in dimension $\ge 5$ all exotic spheres
	can be obtained by using a diffeomorphism of
	$\sphere{n-1}$ to glue two $n$-disks along their boundary.
\end{proposition}

\begin{theorem}[Cerf, \citep{geiges2010eliashberg}]
	Any diffeomorphism of the $3$-sphere $\sphere{3}$
	extends over the $4$-ball $\disk{4}$,
	in other words
	\[
		\Gamma_4 = 0.
	\]
\end{theorem}

\begin{observation}
	Cerf's theorem implies that there are no exotic structures
	on $\sphere{4}$ that can be obtained by gluing
	two $4$-disks along their
	boundary.
\end{observation}





\newpage
% ------------------------------------
\section{Trisections}
% ------------------------------------

\begin{center}
	\textsc{``Trisections are to $4$-manifolds as Heegaard splittings are to
		$3$-manifolds''}
\end{center}

\subsection{References}

\begin{itemize}
	\item Original paper: \citep{gay2016trisecting}
	\item Lecture notes: \citep{gay2019heegaard}
\end{itemize}

\subsection{Definitions}

\marginnote{
	\begin{itemize}
		\item $\#^n A$ is the connected sum of $n$ copies of $A$,
		with $\#^{0} A = \sphere{m}$
		
		\item TODO
		%TODO
	\end{itemize}}
\begin{definition}
	\begin{itemize}
		\item The standard genus $g$ surface is
		\[
			\Sigma_g = {\#}^{g} (\sphere{1} \times \sphere{1})
		\]
		\item The standard genus $g$ solid handlebody is
		\[
			H_g = \natural^g (\sphere{1} \times \disk{2})
		\]
		with $\partial H_{g} = \Sigma_{g}$
		\marginnote{$\partial (A \natural B) = (\partial A) \# (\partial B)$}
		
		\item The standard $4$-dimensional $1$-handlebody
		(of ``genus $k$'') is
		\[
			Z_{k} =\natural^{k} (\sphere{1} \times \disk{3})
		\]
		i.e.\ a $4$-ball to which we attach $k$-many
		$4$-dimensional $1$-handles.
	\end{itemize}
\end{definition}
\begin{marginfigure}
	\begin{center}
		\includegraphics{./pictures/standard_manifolds.png}
	\end{center}
	\caption{Standard manifolds of dimension 2, 3}
	\label{fig:standard_manifolds}
\end{marginfigure}