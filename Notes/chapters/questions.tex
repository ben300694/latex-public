\chapter{Topics to study and Reading List}

% ------------------------------------
\section{Questions}
% ------------------------------------

%TODO
\begin{itemize}
	\item TODO Collect questions that come up, and preferably
	with an answer
\end{itemize}

% ------------------------------------
\section{Reading List}
% ------------------------------------

\begin{itemize}
	\item List of open problems concerning quantum invariants
		is at
		\citep{ohtsuki2002problems}
		
	\item Vassiliev knot invariants, for this
		\citep{bar1995vassiliev}
		
	\item Khovanov homology
		\citep{bar2005khovanov}
	
	\item Kaufman's books, for example
		\citep{kauffman2001knots}
		
	\item Baez's book on gauge theory
		\citep{baez1994gauge}
\end{itemize}


\newpage
% ------------------------------------
\section{Possible Projects}
% ------------------------------------

\subsection{Using Machine Learning to calculate Knot Invariants}

\textbf{Literature:}
\begin{itemize}
	\item \citep{1902.05547}: The authors used a neural network
	to investigate a relationship between the Jones polynomial of a knot
	and the hyperbolic volume of the knot complement in case that the knot is
	hyperbolic
	
	\item \citep{hughes2016neural}
	
	\item \citep{bull2018machine}
\end{itemize}

\noindent \textbf{Ideas:}
\begin{itemize}
	\item Use similar machine learning techniques to approximate
	concordance invariants of knots and links
	
	\item Use Khovanov homology groups as inputs for the
	network in the hope that this captures more combinatorial content
	about the knot than merely the coefficients of knot polynomials.
	One could also try to apply this to other knot homology theories
	
	\item Use a convolutional neural network to encode and process $2$-dimensional input:
	For example the Khovanov and Heegaard-Floer complexes of a knot can be equipped
	with various gradings which could give extra information
	
	\item In the case that there seems to be a previously unknown relationship
	between invariants one could then try to prove this exactly 
\end{itemize}



\subsection{Visualizing Slice Disks in Trisection diagrams}

There is a relative setting where trisections can be used to
describe embedded $2$-dimensional submanifolds of
$4$-manifolds.
Maybe from this one could try to get new statements about
concordance?


\noindent \textbf{TODO:}
\begin{itemize}
	\item Learn more about trisections and in particular about
	representing surfaces in $4$-manifolds
	
	\item Find out what people have already tried with questions
	about concordance using trisections
	
	\item There should be a calculus for modifying trisections?
	Find out more about this.
	Would it be feasible to implement this in a computer program,
	in the spirit of something like Frank Swenton's KirbyCalculator?
	\citep{kirbycalculator}
	
\end{itemize}

