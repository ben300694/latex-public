% % % % % % % % % % % % % % % % % % % % % % % % %
\chapter{Homotopy theory}
% % % % % % % % % % % % % % % % % % % % % % % % %

\section{CW-complexes}

\begin{definition}
	A map $f \colon X \rightarrow Y$ is called a
	\textit{weak homotopy equivalence} \index{homotopy equivalence!weak}
	if it induces isomorphisms
	\[
	\pi_n(X, x_0) \rightarrow \pi_n(Y, f(x_0))
	\]
	for all $n \ge 0$ and all choices of basepoints $x_0$ in $X$.
\end{definition}

\begin{theorem}[Whitehead's Theorem]
	A weak homotopy equivalence between CW-complexes is a homotopy equivalence.
\end{theorem}

% {\cite[Proposition 4.15]{hatcher2002algebraic}}
\begin{proposition}[Geometric interpretation of $n$-connectedness]
	If $(X, A)$ is an $n$-connected CW-pair, then there exists
	a CW-pair $(Z, A) \sim_{\rel A} (X, A)$
	such that all cells of $Z \setminus A$ have dimension greater than $n$.
\end{proposition}

\section{Homology}

\begin{definition}[Acyclic]
	A space $X$ is called \textit{acyclic}\index{acyclic} if $\widetilde{H}_{i}(X) = 0$ for all $i$,
	i.e. if its reduced homology vanishes.
\end{definition}

\begin{example}
	Removing a point from a homology sphere yields an acyclic space.
	If the dimension was at least $3$ this does not change
	the fundamental group, so if we started with a nontrivial homology sphere
	(i.e.\ $\pi_1 \ne 1$) this will give an example of an acyclic, but
	non-contractible space.
	
	This example for the Poincar\'e homology sphere is described in
	\citep[Example 2.38]{hatcher2002algebraic}.
\end{example}