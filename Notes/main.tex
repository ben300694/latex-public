\documentclass[a4paper]{book}

\usepackage[english]{babel}
\usepackage[utf8]{inputenc}
\usepackage[T1]{fontenc}
\usepackage{lmodern}

\usepackage{hyperref, url}

\usepackage{enumerate}

\usepackage{geometry}
\geometry{a4paper, left=2.5cm, right=2cm, top=2cm, bottom=3cm}

\usepackage{amsmath, amssymb, amsfonts, amsthm, mathtools}
\usepackage{faktor}
\usepackage{dsfont}
\usepackage[justification=centering]{caption}

\usepackage{tikz-cd}
\usetikzlibrary{decorations.markings}
\usetikzlibrary{babel}

\usepackage{imakeidx}
\makeindex[columns=2]

%% Environment definitions

\newtheorem{theorem}{Theorem}[section]

\newtheorem{proposition}[theorem]{Proposition}

\newtheorem{lemma}[theorem]{Lemma}

\newtheorem{corollary}[theorem]{Corollary}

\newtheorem{fact}[theorem]{Fact}

\theoremstyle{definition}
\newtheorem{definition}[theorem]{Definition}

\theoremstyle{remark}
\newtheorem{remark}[theorem]{Remark}

\theoremstyle{example}
\newtheorem{example}[theorem]{Example}

\newtheorem{openquestion}{Open question}
\newtheorem*{openquestion*}{Open question}

\newtheorem*{conjecture*}{Conjecture}

%% Math operators

\DeclareMathOperator{\im}{im}
\DeclareMathOperator{\mat}{Mat}
\DeclareMathOperator{\rg}{rg}
\DeclareMathOperator{\id}{id}
\DeclareMathOperator{\gl}{Gl}
\DeclareMathOperator{\spl}{Sl}
\DeclareMathOperator{\cyl}{Cyl}
\DeclareMathOperator{\cone}{Cone}
\DeclareMathOperator{\rel}{rel}

% Macro definitions

\newcommand{\sphere}[1]{\mathbb{S}^{#1}}
\newcommand{\disk}[1]{\mathbb{D}^{#1}}
\newcommand{\interval}{\mathbb{I}}
\newcommand{\RP}[1]{\mathbb{RP}^{#1}}
\newcommand{\CP}[1]{\mathbb{CP}^{#1}}

% \H is already defined
\newcommand{\C}{\mathbb{C}}
\newcommand{\R}{\mathbb{R}}
\newcommand{\Z}{\mathbb{Z}}

\begin{document}

% % % % % % % % % % % % % % % % % % % % % % % % %
\chapter{Homotopy theory}
% % % % % % % % % % % % % % % % % % % % % % % % %

\section{CW-complexes}

\begin{definition}
	A map $f \colon X \rightarrow Y$ is called a
	\textit{weak homotopy equivalence} \index{homotopy equivalence!weak}
	if it induces isomorphisms
	\[
		\pi_n(X, x_0) \rightarrow \pi_n(Y, f(x_0))
	\]
	for all $n \ge 0$ and all choices of basepoints $x_0$ in $X$.
\end{definition}

\begin{theorem}[Whitehead's Theorem]
	A weak homotopy equivalence between CW-complexes is a homotopy equivalence.
\end{theorem}

\begin{proposition}[Geometric interpretation of $n$-connectedness, {\cite[Proposition 4.15]{hatcher2002algebraic}}]
	If $(X, A)$ is an $n$-connected CW-pair, then there exists
	a CW-pair $(Z, A) \sim_{\rel A} (X, A)$
	such that all cells of $Z \setminus A$ have dimension greater than $n$.
\end{proposition}

\section{Homology}

\begin{definition}[Acyclic]
	A space $X$ is called \textit{acyclic}\index{acyclic} if $\widetilde{H}_{i}(X) = 0$ for all $i$,
	i.e. if its reduced homology vanishes.
\end{definition}

\begin{example}
	Removing a point from a homology sphere yields an acyclic space.
	This example for the Poincare homology sphere is described in
	\cite[Example 2.38]{hatcher2002algebraic}
	TODO Insert proof. %TODO
\end{example}


% % % % % % % % % % % % % % % % % % % % % % % % %
\chapter{Knot Theory}
% % % % % % % % % % % % % % % % % % % % % % % % %

\section{Definitions}

\begin{definition}
	If $K$ is an oriented knot, then
	\begin{itemize}
		\item the \textit{reverse}\index{knot!reverse} $\overline{K}$
		is $K$ with the opposite orientation
		
		\item the \textit{obverse}\index{knot!obverse} $rK$ is
		the reflection of $K$ in a plane
		
		\item the \textit{inverse}\index{knot!inverse} $r \overline{K}$
		is the concordance inverse of $K$.
	\end{itemize}
\end{definition}

\begin{proposition}
	For $K \subset \sphere{3}$ we have that
	$K \# r \overline{K}$ is slice, even ribbon.
\end{proposition}


\subsection{Alexander polynomial}

\begin{definition}
	$L$ oriented link with Seifert matrix $A$, then the first homology of
	the infinite cyclic covering of the link complement, $H_1(X_{\infty} ; \Z)$,
	has square presentation matrix $t A - A^{T}$.
	
	The \textit{Alexander polynomial}\index{Alexander!polynomial} of $L$ is given by
	\begin{equation*}
		\Delta_{L}(t) \doteq \det(t A - A^{T})
	\end{equation*}
	where $\doteq$ means ``up to a multiplication with a unit $\{ \pm t^{\pm n} \}$
	of the Laurent ring $\Z[t, t^{-1}]$''.
\end{definition}

\begin{remark}
	$\Z[t^{\pm 1}]$ is \textbf{not} a PID.
\end{remark}


\subsection{Invariants}

\begin{definition}
	The tunnel number $t(K)$ of a knot $K \subset \sphere{3}$ is the minimal number of arcs
	that must be added to the knot (forming a graph with three edges at a vertex) so that
	its complement in $\sphere{3}$ is a handlebody. The same definition is
	valid for links. \\
	The boundary will be a minimal Heegaard splitting of the knot complement
	(The knot complement is a manifold with boundary, so what is the definition
	of a Heegard splitting in that case?).
\end{definition}

\begin{remark}
	Every link has a tunnel number, this can be seen by adding a ``vertical''
	tunnel at every crossing in a link diagram.
	This shows that the tunnel number of a knot is always less than or equal
	to the crossing number, $t(K) \le c(K)$.
\end{remark}

\begin{example}
	\begin{itemize}
		\item The unknot is the only knot with tunnel number 0. (Why?)
		\item The trefoil knot has tunnel number 1.
		\item The figure eight knot has tunnel number 1.
	\end{itemize}
\end{example}

\section{Open questions}

\begin{openquestion}
	Is the crossing number of a satellite knot bigger than that of its companion?
\end{openquestion}
	
% % % % % % % % % % % % % % % % % % % % % % % % %
\chapter{4-manifolds}
% % % % % % % % % % % % % % % % % % % % % % % % %

\bibliography{mybib}{}
\bibliographystyle{alpha}

\printindex

\end{document}
